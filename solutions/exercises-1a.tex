%! Author = cs-cordero
%! Date = 9/28/22

% Preamble
\documentclass[11pt]{article}

% Packages
\usepackage{amsmath}
\usepackage{titling}
\usepackage{amssymb}

\title{Linear Algebra Done Right, Third Edition \\ \vskip 0.5em \large Exercises 1.A}
\preauthor{}
\author{}
\postauthor{}
\predate{}
\postdate{}
\date{}

% Custom commands
\newcommand{\problemseparator}{\vskip 1em \hrule \vskip 1em}
\newcommand{\finishsolution}{\rightline{$\blacksquare$}}

% Document
\begin{document}
    \maketitle
    \subsection*{Problem 1}
    Suppose $a$ and $b$ are real numbers, not both 0.
    Find real numbers $c$ and $d$ such that:
    \[ \frac{1}{a + bi} = c + di \]

    \problemseparator

    Using complex number arithmetic, we can solve for $c$ and $d$ in terms of $a$ and $b$ with the following:

    \begin{equation*}
        \begin{split}
            \begin{aligned}
                \frac{1}{a + bi} &= c + di \\
                1 &= (a + bi)(c + di) \\
                1 + 0i &= ac + adi + bci + bdi^2 \\
                1 + 0i &= (ac - bd) + (ad + bc)i
            \end{aligned}
        \end{split}
        \qquad \qquad
        \begin{split}
            \begin{aligned}
                1 &= ac - bd \\
                1 + bd &= ac \\
                \frac{1 + bd}{a} &= c
            \end{aligned}
        \end{split}
        \qquad \qquad
        \begin{split}
            \begin{aligned}
                0 &= ad + bc \\
                -ad &= bc \\
                \frac{-ad}{b} &= c
            \end{aligned}
        \end{split}
    \end{equation*}

    \begin{equation*}
        \begin{split}
            \begin{aligned}
                \frac{1 + bd}{a} &= \frac{-ad}{b} \\
                b + b^{2}d &= -a^{2}d \\
                a^{2}d + b^{2}d &= -b \\
                d &= \frac{-b}{a^2 + b^2}
            \end{aligned}
        \end{split}
        \qquad \qquad
        \begin{split}
            \begin{aligned}
                c &= \frac{-ad}{b} \\
                c &= \frac{-a(\frac{-b}{a^2 + b^2})}{b} \\
                c &= \frac{\frac{ab}{a^2 + b^2}}{b} \\
                c &= \frac{a}{a^2 + b^2}
            \end{aligned}
        \end{split}
    \end{equation*}

    \finishsolution
    \newpage

    \subsection*{Problem 2}
    Show that:
    \[ \frac{-1 + \sqrt{3}i}{2} \]
    \ldots is a cube root of 1 (meaning that its cube equals 1).

    \problemseparator

    We just have to use our basic arithmetic to demonstrate that cubing the term results in giving us $1$.

    \[
        \begin{aligned}
            1 &= (\frac{-1 + \sqrt{3}i}{2})^{3} \\
            &= \frac{(-1 + \sqrt{3}i)^{3}}{2^3} \\
            &= \frac{(-1 + \sqrt{3}i)(-1 + \sqrt{3}i)(-1 + \sqrt{3}i)}{8} \\
            &= \frac{(1 - 2\sqrt{3}i + 3i^2)(-1 + \sqrt{3}i)}{8} \\
            &= \frac{(1 - 2\sqrt{3}i - 3)(-1 + \sqrt{3}i)}{8} \\
            &= \frac{(-2 - 2\sqrt{3}i)(-1 + \sqrt{3}i)}{8} \\
            &= \frac{2 - 2\sqrt{3}i + 2\sqrt{3}i - 2(3)i^2}{8} \\
            &= \frac{2 - 6i^2}{8} \\
            &= \frac{2 + 6}{8} \\
            &= \frac{8}{8} \\
            &= 1
        \end{aligned}
    \]

    \finishsolution
    \newpage

    \subsection*{Problem 3}
    Find two distinct square roots of $i$.

    \problemseparator

    A square root of $i$ is one where for some $x$, $x^2 = i$.
    Since we're working with imaginary $i$, let's suppose x is a complex number, such that $a, b \in \mathbb{R}$ and:
    \[ (a + bi)^2 = i\].

    Then it follows that:

    \[
        \begin{aligned}
            i &= (a + bi)^2 \\
            0 + 1i &= (a + bi)(a + bi) \\
            0 + 1i &= (a^2 - b^2) + 2abi
        \end{aligned}
    \]

    We now have:

    \[
        \begin{split}
            \begin{aligned}
                0 &= a^2 - b^2 \\
                b^2 &= a^2 \\
                b &= \pm a
            \end{aligned}
        \end{split}
        \qquad \rightrightarrows \qquad
        \begin{split}
            \begin{aligned}
                1 &= 2ab \\
                1 &= 2a(a) \\
                1 &= 2a^2 \\
                \pm \frac{1}{\sqrt{2}} &= a
            \end{aligned}
        \end{split}
        \qquad \qquad
        \begin{split}
            \begin{aligned}
                1 &= 2ab \\
                1 &= 2a(-a) \\
                \frac{1}{2} &= -a^2 \\
                &\text{This branch is impossible}
            \end{aligned}
        \end{split}
    \]

    The first equation leads to the second and third equations above.
    The third, however is a dead-end, because it leads to a contradiction.
    This therefore implies that despite the first equation finding $b = \pm a$, it must be that $b = a$.
    \
    The second equation's branching path can be used to determine two possibilities for $b$.

    \[
        \begin{split}
            \begin{aligned}
                b &= \frac{1}{2\left(\frac{1}{\sqrt{2}}\right)} \\
                b &= \frac{\sqrt{2}}{2}
            \end{aligned}
        \end{split}
        \qquad \qquad
        \begin{split}
            \begin{aligned}
                b &= \frac{1}{2\left(-\frac{1}{\sqrt{2}}\right)} \\
                b &= -\frac{\sqrt{2}}{2}
            \end{aligned}
        \end{split}
    \]

    Therefore, two distinct square roots of $i$ are:

    \[
        \begin{split}
            \frac{1}{\sqrt{2}} + \frac{\sqrt{2}}{2}i
        \end{split}
        \qquad \qquad
        \begin{split}
            -\frac{1}{\sqrt{2}} - \frac{\sqrt{2}}{2}i
        \end{split}
    \]

    \finishsolution
    \newpage

    \subsection*{Problem 4}
    Show that $\alpha + \beta = \beta + \alpha$ for all $\alpha, \beta \in \mathbb{C}$.

    \problemseparator

    This can be demonstrated using complex number arithmetic.

    Suppose that:

    \begin{align*}
        \alpha &= a + bi \\
        \beta &= c + di
    \end{align*}

    Then it follows that:

    \[
        \begin{aligned}
            \alpha + \beta &= \beta + \alpha \\
            (a + bi) + (c + di) &= (c + di) + (a + bi) \\
            (a + c) + (b + d)i &= (c + a) + (d + b)i \\
            (a + c) + (b + d)i &= (a + c) + (b + d)i
        \end{aligned}
    \]

    \finishsolution

    \subsection*{Problem 5}
    Show that $(\alpha + \beta) + \lambda = \alpha + (\beta + \lambda)$ for all $\alpha, \beta, \lambda \in \mathbb{C}$.

    \problemseparator

    This can be demonstrated using complex number arithmetic.

    Suppose that:

    \begin{align*}
        \alpha &= a + bi \\
        \beta &= c + di \\
        \lambda &= e + fi
    \end{align*}

    Then it follows that:

    \[
        \begin{aligned}
            (\alpha + \beta) + \lambda &= \alpha + (\beta + \lambda) \\
            ((a + bi) + (c + di)) + (e + fi) &= (a + bi) + ((c + di) + (e + fi)) \\
            ((a + c) + (b + d)i) + (e + fi) &= (a + bi) + ((c + e) + (d + f)i) \\
            (a + c + e) + (b + d + f)i &= (a + c + e) + (b + d + f)i
        \end{aligned}
    \]

    \finishsolution
    \newpage

    \subsection*{Problem 6}
    Show that $(\alpha \beta) \lambda = \alpha (\beta \lambda)$ for all $\alpha, \beta, \lambda \in \mathbb{C}$.

    \problemseparator
    This can be demonstrated using complex number arithmetic.

    Suppose that:

    \begin{align*}
        \alpha &= a + bi \\
        \beta &= c + di \\
        \lambda &= e + fi
    \end{align*}

    Then it follows that:

    \[
        \begin{aligned}
            (\alpha \beta) \lambda &= \alpha (\beta \lambda) \\
            ((a + bi)(c + di))(e + fi) &= (a + bi)((c + di)(e + fi)) \\
            ((ac - bd) + (ad + bc)i)(e + fi) &= (a + bi)((ce - df) + (cf + de)i)
        \end{aligned}
    \]
    \begin{multline*}
        (ace - bde - adf - bcf) + (acf - bdf + ade + bce)i \\
            = (ace - adf - bcf - bde) + (acf + ade + bce - bdf)i
    \end{multline*}
    \begin{multline*}
        (ace - bde - adf - bcf) + (acf - bdf + ade + bce)i \\
            = (ace - bde - adf - bcf) + (acf - bdf + ade + bce)i
    \end{multline*}

    \finishsolution
    \newpage

    \subsection*{Problem 7}
    Show that for every $\alpha \in \mathbb{C}$, there exists a unique $\beta \in \mathbb{C}$ such that $\alpha + \beta = 0$.

    \problemseparator
    This can be demonstrated using complex number arithmetic.

    Define $\beta$ as:

    \[ \alpha + \beta = 0 \]

    \ldots such that $\alpha = a + bi$ and $\beta = c + di$, and $a, b, c, d \in \mathbb{R}$.

    Then it follows that:

    \[
        \begin{aligned}
            \alpha + \beta &= 0 \\
            (a + bi) + (c + di) &= 0 + 0i \\
            (a + c) + (b + d)i &= 0 + 0i
        \end{aligned}
    \]
    \[
        \begin{split}
            a + c = 0
        \end{split}
        \qquad \qquad
        \begin{split}
            b + d = 0
        \end{split}
    \]

    Thus, for every $a$ there is only a single solution, $-c$ that will make $a + c = 0$.
    And for every $b$ there is only a single solution, $-d$ that will make $b + d = 0$.

    Therefore, for every $\alpha = a + bi$, there exists a single $c + di = \beta$ that will make $\alpha + \beta = 0$.

    \finishsolution
    \newpage

    \subsection*{Problem 8}
    Show that for every $\alpha \in \mathbb{C}$ with $\alpha \neq 0$, there exists a unique $\beta \in \mathbb{C}$ such that $\alpha \beta = 1$.

    \problemseparator
    This can be demonstrated using complex number arithmetic.

    Define $\beta$ as:

    \[ \alpha \beta = 1 \]

    \ldots such that $\alpha = a + bi$ and $\beta = c + di$, and $a, b, c, d \in \mathbb{R}$.

    Then it follows that:

    \[
        \begin{aligned}
            \alpha \beta &= 1 \\
            (a + bi)(c + di) &= 1 + 0i \\
            (ac - bd) + (ad + bc)i &= 1 + 0i
        \end{aligned}
    \]
    \vskip 1em
    \[
        \begin{split}
            \begin{aligned}
                ac - bd &= 1 \\
                c &= \frac{1 + bd}{a}
            \end{aligned}
        \end{split}
        \qquad \qquad
        \begin{split}
            \begin{aligned}
                ad + bc &= 0 \\
                -\frac{ad}{b} &= c
            \end{aligned}
        \end{split}
    \]
    \vskip 1em
    \[
        \begin{split}
            \begin{aligned}
                \frac{1 + bd}{a} &= -\frac{ad}{b} \\
                b + b^{2}d &= -a^{2}d \\
                a^{2} + b^{2}d &= -b \\
                d &= \frac{-b}{a^2 + b^2}
            \end{aligned}
        \end{split}
        \qquad \qquad
        \begin{split}
            \begin{aligned}
                c &= -\frac{ad}{b} \\
                c &= -\frac{a(\frac{-b}{a^2 + b^2})}{b} \\
                c &= \frac{\frac{ab}{a^2 + b^2}}{b} \\
                c &= \frac{a}{a^2 + b^2}
            \end{aligned}
        \end{split}
    \]

    Thus, we have shown that for any arbitrary $\alpha = a + bi$, there is only one $c$ and $d$ for $\beta = c + di$ such that $\alpha \beta = 1$:

    \[ \beta = \frac{a}{a^2 + b^2} + \frac{-b}{a^2 + b^2}i \].

    \finishsolution
    \newpage

    \subsection*{Problem 9}
    Show that for every $\lambda(\alpha + \beta) = \lambda \alpha + \lambda \beta$ for all $\lambda, \alpha, \beta \in \mathbb{C}$.

    \problemseparator
    This can be demonstrated using complex number arithmetic.

    Suppose that:

    \begin{align*}
        \alpha &= a + bi \\
        \beta &= c + di \\
        \lambda &= x + yi
    \end{align*}

    Then it follows that:

    \[
        \begin{aligned}
            \lambda (\alpha + \beta) &= \lambda \alpha + \lambda \beta \\
            (x + yi) ((a + bi) + (c +di)) &= (x + yi)(a + bi) + (x + yi)(c + di) \\
        \end{aligned}
    \]
    \begin{multline*}
        (x + yi) ((a + c) + (b + d)i) \\
            = ((ax - by) + (bx + ay)i) + ((cx - dy) + (dx + cy)i)
    \end{multline*}
    \begin{multline*}
        (ax + cx - by - dy) + (bx + dx + ay + cy)i \\
            = (ax - by + cx - dy) + (bx + ay + dx + cy)i
    \end{multline*}
    \begin{multline*}
        (ax + cx - by - dy) + (bx + dx + ay + cy)i \\
            = (ax + cx - by - dy) + (bx + dx + ay + cy)i
    \end{multline*}

    \finishsolution
    \newpage

    \subsection*{Problem 10}
    Find $x \in \mathbb{R}^4$ such that

    \[ (4, -3, 1, 7) + 2x = (5, 9, -6, 8) \]

    \problemseparator
    This can be demonstrated using complex number arithmetic.
    Adding two lists are done component-wise.

    \[
        \begin{aligned}
            (4, -3, 1, 7) + 2x &= (5, 9, -6, 8) \\
            (4, -3, 1, 7) + 2(x_1, x_2, x_3, x_4) &= (5, 9, -6, 8)
        \end{aligned}
    \]
    \vskip 1em
    \[
        \begin{split}
            \begin{aligned}
                4 + 2x_1 &= 5 \\
                -3 + 2x_2 &= 9 \\
                1 + 2x_3 &= -6 \\
                7 + 2x_4 &= 8
            \end{aligned}
        \end{split}
        \qquad \implies \qquad
        \begin{split}
            \begin{aligned}
                x_1 &= \frac{1}{2} \\
                x_2 &= 6 \\
                x_3 &= -\frac{7}{2} \\
                x_4 &= \frac{1}{2}
            \end{aligned}
        \end{split}
    \]

    Thus, $x = \left(\frac{1}{2}, 6, -\frac{7}{2}, \frac{1}{2} \right)$.

    \finishsolution
    \newpage

    \subsection*{Problem 11}
    Explain why there does not exist $\lambda \in \mathbb{C}$ such that:

    \[ \lambda (2 - 3i, 5 + 4i, -6 + 7i) = (12 - 5i, 7 + 22i, -32 - 9i) \]

    \problemseparator

    Define $\lambda$ as $\lambda = a + bi$ such that $a, b \in \mathbb{R}$.

    In order for the equation above to hold, there must be a single $\lambda$ such that all of the following component-wise equations also hold:

    \[
        \begin{aligned}
            (a + bi)(2 - 3i) &= 12 - 5i \\
            (a + bi)(5 + 4i) &= 7 + 22i \\
            (a + bi)(-6 + 7i) &= -32 - 9i
        \end{aligned}
    \]

    It can be proven that there exists one $a + bi = \lambda$ such that the first two equations hold.
    This value is $\lambda = 3 + 2i$.

    \[
        \begin{split}
            \begin{aligned}
                (3 + 2i)(2 - 3i) &= 12 - 5i \\
                6 - 9i + 4i - 6i^2 &= 12 - 5i \\
                12 - 5i &= 12 - 5i
            \end{aligned}
        \end{split}
        \qquad \qquad
        \begin{split}
            \begin{aligned}
                (3 + 2i)(5 + 4i) &= 7 + 22i \\
                15 + 12i + 10i + 8i^2 &= 7 + 22i \\
                7 + 22i &= 7 + 22i
            \end{aligned}
        \end{split}
    \]

    However, when applied to the third equation, it does not hold, which means that the $\lambda$ that fulfills the target equation does not exist.

    \[
        \begin{aligned}
            (3 + 2i)(-6 + 7i) &= -32 - 9i \\
            -18 + 21i - 12i + 14i^2 &= -32 - 9i \\
            -32 + 9i &\neq -32 - 9i
        \end{aligned}
    \]

    \finishsolution
    \newpage

    \subsection*{Problem 12}
    Show that $(x + y) + z = x + (y + z)$ for all $x, y, z \in \mathbb{F}^n$.

    \problemseparator

    Suppose that:

    \[
        \begin{aligned}
            x &= x_1, \cdots, x_n \\
            y &= y_1, \cdots, y_n \\
            z &= z_1, \cdots, z_n
        \end{aligned}
    \]

    Then it follows that:

    \[
        \begin{aligned}
            (x + y) + z &= ((x_1, \cdots, x_n) + (y_1, \cdots, y_n)) + (z_1, \cdots, z_n) \\
                &= (x_1 + y_1, \cdots, x_n + y_n) + (z_1, \cdots, z_n) \\
                &= x_1 + y_1 + z_1, \cdots, x_n + y_n + z_n \\
                &= x_1 + (y_1 + z_1), \cdots, x_n + (y_n + z_n) \\
                &= (x_1, \cdots, x_n) + (y_1 + z_1, \cdots, y_n + z_n) \\
                &= x + (y + z)
        \end{aligned}
    \]


    \finishsolution

    \subsection*{Problem 13}
    Show that $(ab)x = a(bx)$ for all $x \in \mathbb{F}^n$ and all $a, b \in \mathbb{F}$.

    \problemseparator

    Suppose that:

    \[ x = (x_1, \cdots, x_n) \]

    Then it follows that:

    \[
        \begin{aligned}
            (ab)x &= (ab)(x_1, \cdots, x_n) \\
                &= abx_1, \cdots, abx_n \\
                &= a(bx_1), \cdots, a(bx_n) \\
                &= a\left(bx_1, \cdots, bx_n\right) \\
                &= a(bx)
        \end{aligned}
    \]

    \finishsolution
    \newpage

    \subsection*{Problem 14}
    Show that $1x = x$ for all $x \in \mathbb{F}^n$.

    \problemseparator

    Suppose that:

    \[ x = (x_1, \cdots, x_n) \]

    Then it follows that:

    \[
        \begin{aligned}
            1x &= 1(x_1, \cdots, x_n) \\
                &= 1x_1, \cdots, 1x_n \\
                &= x_1, \cdots, x_n \\
                &= x
        \end{aligned}
    \]

    \finishsolution

    \subsection*{Problem 15}
    Show that $\lambda (x + y) = \lambda x + \lambda y$ for all $\lambda \in \mathbb{F}$ and all $x, y \in \mathbb{F}^n$.

    \problemseparator

    Suppose that:

    \[
        \begin{aligned}
            x = (x_1, \cdots, x_n) \\
            y = (y_1, \cdots, y_n)
        \end{aligned}
    \]

    Then it follows that:

    \[
        \begin{aligned}
            \lambda (x + y) &= \lambda ((x_1, \cdots, x_n) + (y_1, \cdots, y_n)) \\
            &= \lambda (x_1 + y_1, \cdots, x_n + x_n) \\
            &= \lambda (x_1 + y_1), \cdots, \lambda (x_n + x_n) \\
            &= \lambda x_1 + \lambda y_1, \cdots, \lambda x_n + \lambda x_n \\
            &= (\lambda x_1, \cdots, \lambda x_n) + (\lambda y_1, \cdots, \lambda y_n) \\
            &= \lambda (x_1, \cdots, x_n) + \lambda (y_1, \cdots, y_n) \\
            &= \lambda x + \lambda y
        \end{aligned}
    \]

    \finishsolution
    \newpage
    \subsection*{Problem 16}
    Show that $(a + b)x = ax + bx$ for all $a, b \in \mathbb{F}$ and all $x \in \mathbb{F}^n$.

    \problemseparator

    Suppose that:

    \[ x = (x_1, \cdots, x_n) \]

    Then it follows that:

    \[
        \begin{aligned}
            (a + b)x &= (a + b)(x_1, \cdots, x_n) \\
            &= (a + b)x_1, \cdots, (a + b)x_n \\
            &= ax_1 + bx_1, \cdots, ax_n + bx_n \\
            &= (ax_1, \cdots, ax_n) + (bx_1, \cdots, bx_n) \\
            &= ax + bx
        \end{aligned}
    \]

    \finishsolution
\end{document}