%! Author = cs-cordero
%! Date = 9/29/22

% Preamble
\documentclass[11pt]{article}

% Packages
\usepackage{amsmath}
\usepackage{titling}
\usepackage{amssymb}

\title{Linear Algebra Done Right, Third Edition \\ \vskip 0.5em \large Exercises 1.B}
\preauthor{}
\author{}
\postauthor{}
\predate{}
\postdate{}
\date{}

% Custom commands
\newcommand{\problemseparator}{\vskip 1em \hrule \vskip 1em}
\newcommand{\finishsolution}{\rightline{$\blacksquare$}}

% Document
\begin{document}
    \maketitle
    \subsection*{Problem 1}
    Prove that $-(-v) = v$ for every $v \in \mathbb{V}$.

    \problemseparator

    First, the notation $-v$ denotes a scalar multiplication of $(-1)v$.

    Therefore:

    \[
        \begin{aligned}
            -(-v) &= -1(-1(v)) \\
                &= 1(v) \\
                &= v
        \end{aligned}
    \]

    \finishsolution
    \newpage

    \subsection*{Problem 2}
    Suppose $a \in \mathbb{F}, v \in \mathbb{V}$, and $av = 0$.
    Prove that $a = 0$ or $v = 0$.

    \problemseparator

    First, let's suppose $v \neq 0$ and $a = 0$.
    Note that any element, $f$, of a field, $\mathbb{F}$, has a multiplicative inverse, such that $ff^{-1} = 1$.
    We can multiply both sides of the equation by this multiplicative inverse.
    It then follows that:

    \[
        \begin{aligned}
            0 &= av \\
            0v^{-1} &= avv^{-1} \\
            0 &= a(vv^{-1}) \\
            0 &= a1 \\
            0 &= a
        \end{aligned}
    \]

    Therefore, $a$ must be zero.

    If we suppose $a \neq 0$ and $v = 0$, the outcome is similar.

    \[
        \begin{aligned}
            0 &= av \\
            0a^{-1} &= aa^{-1}v \\
            0 &= (aa^{-1})v \\
            0 &= 1v \\
            0 &= v
        \end{aligned}
    \]

    Thus $v$ must be zero.

    Finally, it should be obvious to see that if \emph{both} $a = 0$ and $v = 0$, then $av = 0$.

    \finishsolution
    \newpage

    \subsection*{Problem 3}
    Suppose $v, w \in \mathbb{V}$.
    Explain why there exists a unique $x \in \mathbb{V}$ such that $v + 3x = w$.

    \problemseparator

    \[
        \begin{aligned}
            v + 3x &= w \\
            v + 3x - w &= 0 \\
            v - w + 3x &= 0 \\
            (v - w) + 3x &= 0
        \end{aligned}
    \]

    Since $v - w = v + (-w) \in \mathbb{V}$ due to the definition of addition for the vector space $\mathbb{V}$,
    we can define $v' = (v - w)$, and so we have $v' + 3x = 0$.

    By theorem 1.26, which states that ``every element in a vector space has a unique additive inverse'', the
    additive inverse of $v'$ is $3x$ and it must be unique.

    Therefore, there is a unique $x$ such that $v + 3x = w$.

    \finishsolution
    \newpage

    \subsection*{Problem 4}
    The empty set is not a vector space.
    The empty set fails to satisfy only one of the requirements listed in 1.19.
    Which one?

    \subsubsection*{1.19 Definition \emph{vector space}}
    A \emph{vector space} is a set $\mathbb{V}$ along with an addition on $\mathbb{V}$ and a scalar multiplication on $\mathbb{V}$ such that the following properties hold:

    \vskip 0.5em \noindent{\textbf{commutativity}}

    $u + v = v + u$ for all $u, v \in \mathbb{V}$.

    \vskip 0.5em \noindent{\textbf{associativity}}

    $(u + v) + w = u + (v + w)$ and $(ab)v = a(bv)$ for all $u, v, w \in \mathbb{V}$ and all $a, b \in \mathbb{F}$.

    \vskip 0.5em \noindent{\textbf{additive identity}}

    there exists an element $0 \in \mathbb{V}$ such that $v + 0 = v$ for all $v \in \mathbb{V}$.

    \vskip 0.5em \noindent{\textbf{additive inverse}}

    for every $v \in \mathbb{V}$, there exists $w \in \mathbb{V}$ such that $v + w = 0$.

    \vskip 0.5em \noindent{\textbf{multiplicative identity}}

    $1v = v$ for all $v \in \mathbb{V}$

    \vskip 0.5em \noindent{\textbf{distributive properties}}

    $a(u + v) = au + av$ and $(a + b)v = av + bv$ for all $a, b \in \mathbb{F}$ and all $u, v \in \mathbb{F}$.

    \problemseparator

    For the empty set, ${}$, the properties of \emph{commutativity}, \emph{associativity},
    \emph{additive inverse}, \emph{multiplicative identity}, and \emph{distributive properties}
    are known to be \textbf{vacuously true}.

    The property it fails to meet, is the \emph{additive identity}, because in the empty set,
    an element called zero $0$ does not exist.

    \finishsolution
    \newpage

    \subsection*{Problem 5}
    Show that the definition of a vector space (1.19, see copied statement in problem 4 above), the additive inverse condition can be replaced with teh condition that

    \[ 0v = 0 \text{ for all } v \in \mathbb{V} \]

    Here the 0 on the left side is the number 0, and the 0 on the right side is the additive identity of $\mathbb{V}$.
    (The phrase ``a condition can be replaced'' in a definition means that the collection of objects satisfying the definition is unchanged if the original condition is replaced with the new condition.)

    \problemseparator

    If the above condition could replace the additive inverse condition,
    it follows that $0v = 0$ if and only if there exists $w \in \mathbb{V}$ such that $v + w = 0$

    \[
        \begin{split}
            0v = v
        \end{split}
        \qquad \qquad
        \begin{split}
            v + w = 0
        \end{split}
    \]

    In both of the above equations, the $0$ on the right-hand sides are both referring to the same value: the \emph{additive identity} of $\mathbb{V}$.
    Therefore, we have:

    \[
        \begin{aligned}
            0v &= v + w \\
            0v - v &= w \\
            v(0 - 1) &= w \\
            -v &= w \\
            0 &= v + w
        \end{aligned}
    \]


    \finishsolution
    \newpage

    \subsection*{Problem 6}
    Let $\infty$ and $-\infty$ denote two distinct objects, neither of which is in $\mathbb{R}$.
    Define an addition and scalar multiplication on $\mathbb{R} \cup \left\{ \infty \right\} \cup \left\{ -\infty \right\}$ as you could guess from the notation.
    Specifically, the sum and product of two real numbers is as usual, and for $t \in \mathbb{R}$ define

    \[
        \begin{split}
            t \infty = \left\{
                \begin{array}{lr}
                    -\infty & \text{if } t < 0, \\
                    0 & \text{if } t = 0, \\
                    \infty & \text{if } t > 0
                \end{array}
                \right.
        \end{split}
        \qquad \qquad
        \begin{split}
            t(-\infty) = \left\{
            \begin{array}{lr}
                \infty & \text{if } t < 0, \\
                0 & \text{if } t = 0, \\
                -\infty & \text{if } t > 0
            \end{array}
            \right.
        \end{split}
    \]
    \[
        \begin{split}
            t + \infty = \infty + t = \infty,
        \end{split}
        \qquad \qquad
        \begin{split}
            t + (-\infty) = (-\infty) + t = -\infty,
        \end{split}
    \]
    \[
        \begin{split}
            \infty + \infty = \infty,
        \end{split}
        \qquad \qquad
        \begin{split}
            (-\infty) + (-\infty) = -\infty,
        \end{split}
        \qquad \qquad
        \begin{split}
            \infty + (-\infty) = 0
        \end{split}
    \]

    \noindent{Is $\mathbb{R} \cup \left\{\infty\right\} \cup \left\{-\infty\right\}$ a vector space over $\mathbb{R}$? Explain.}

    \problemseparator

    $\mathbb{R} \cup \left\{\infty\right\} \cup \left\{-\infty\right\}$ is \textbf{not} a vector space over $\mathbb{R}$.
    It violates the property of associativity.

    If it were a vector space, the following equation would hold, but it doesn't.

    \[
        \begin{aligned}
            \left( t + \infty \right) - \infty &= t + \left( \infty - \infty \right) \\
            \infty - \infty &= t + 0 \\
            0 &\neq t
        \end{aligned}
    \]

    Hence, $\left( t + \infty \right) - \infty$ does not equal $t + \left( \infty - \infty \right)$, which
    contradicts the definition of a vector space and precludes this set from being one.

    \finishsolution
    \newpage
\end{document}